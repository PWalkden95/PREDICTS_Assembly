% Options for packages loaded elsewhere
\PassOptionsToPackage{unicode}{hyperref}
\PassOptionsToPackage{hyphens}{url}
%
\documentclass[
]{article}
\usepackage{amsmath,amssymb}
\usepackage{lmodern}
\usepackage{iftex}
\ifPDFTeX
  \usepackage[T1]{fontenc}
  \usepackage[utf8]{inputenc}
  \usepackage{textcomp} % provide euro and other symbols
\else % if luatex or xetex
  \usepackage{unicode-math}
  \defaultfontfeatures{Scale=MatchLowercase}
  \defaultfontfeatures[\rmfamily]{Ligatures=TeX,Scale=1}
\fi
% Use upquote if available, for straight quotes in verbatim environments
\IfFileExists{upquote.sty}{\usepackage{upquote}}{}
\IfFileExists{microtype.sty}{% use microtype if available
  \usepackage[]{microtype}
  \UseMicrotypeSet[protrusion]{basicmath} % disable protrusion for tt fonts
}{}
\makeatletter
\@ifundefined{KOMAClassName}{% if non-KOMA class
  \IfFileExists{parskip.sty}{%
    \usepackage{parskip}
  }{% else
    \setlength{\parindent}{0pt}
    \setlength{\parskip}{6pt plus 2pt minus 1pt}}
}{% if KOMA class
  \KOMAoptions{parskip=half}}
\makeatother
\usepackage{xcolor}
\IfFileExists{xurl.sty}{\usepackage{xurl}}{} % add URL line breaks if available
\IfFileExists{bookmark.sty}{\usepackage{bookmark}}{\usepackage{hyperref}}
\hypersetup{
  pdftitle={Internal\_hypervolume\_structure},
  pdfauthor={Patrick Alexander Walkden},
  hidelinks,
  pdfcreator={LaTeX via pandoc}}
\urlstyle{same} % disable monospaced font for URLs
\usepackage[margin=1in]{geometry}
\usepackage{color}
\usepackage{fancyvrb}
\newcommand{\VerbBar}{|}
\newcommand{\VERB}{\Verb[commandchars=\\\{\}]}
\DefineVerbatimEnvironment{Highlighting}{Verbatim}{commandchars=\\\{\}}
% Add ',fontsize=\small' for more characters per line
\usepackage{framed}
\definecolor{shadecolor}{RGB}{248,248,248}
\newenvironment{Shaded}{\begin{snugshade}}{\end{snugshade}}
\newcommand{\AlertTok}[1]{\textcolor[rgb]{0.94,0.16,0.16}{#1}}
\newcommand{\AnnotationTok}[1]{\textcolor[rgb]{0.56,0.35,0.01}{\textbf{\textit{#1}}}}
\newcommand{\AttributeTok}[1]{\textcolor[rgb]{0.77,0.63,0.00}{#1}}
\newcommand{\BaseNTok}[1]{\textcolor[rgb]{0.00,0.00,0.81}{#1}}
\newcommand{\BuiltInTok}[1]{#1}
\newcommand{\CharTok}[1]{\textcolor[rgb]{0.31,0.60,0.02}{#1}}
\newcommand{\CommentTok}[1]{\textcolor[rgb]{0.56,0.35,0.01}{\textit{#1}}}
\newcommand{\CommentVarTok}[1]{\textcolor[rgb]{0.56,0.35,0.01}{\textbf{\textit{#1}}}}
\newcommand{\ConstantTok}[1]{\textcolor[rgb]{0.00,0.00,0.00}{#1}}
\newcommand{\ControlFlowTok}[1]{\textcolor[rgb]{0.13,0.29,0.53}{\textbf{#1}}}
\newcommand{\DataTypeTok}[1]{\textcolor[rgb]{0.13,0.29,0.53}{#1}}
\newcommand{\DecValTok}[1]{\textcolor[rgb]{0.00,0.00,0.81}{#1}}
\newcommand{\DocumentationTok}[1]{\textcolor[rgb]{0.56,0.35,0.01}{\textbf{\textit{#1}}}}
\newcommand{\ErrorTok}[1]{\textcolor[rgb]{0.64,0.00,0.00}{\textbf{#1}}}
\newcommand{\ExtensionTok}[1]{#1}
\newcommand{\FloatTok}[1]{\textcolor[rgb]{0.00,0.00,0.81}{#1}}
\newcommand{\FunctionTok}[1]{\textcolor[rgb]{0.00,0.00,0.00}{#1}}
\newcommand{\ImportTok}[1]{#1}
\newcommand{\InformationTok}[1]{\textcolor[rgb]{0.56,0.35,0.01}{\textbf{\textit{#1}}}}
\newcommand{\KeywordTok}[1]{\textcolor[rgb]{0.13,0.29,0.53}{\textbf{#1}}}
\newcommand{\NormalTok}[1]{#1}
\newcommand{\OperatorTok}[1]{\textcolor[rgb]{0.81,0.36,0.00}{\textbf{#1}}}
\newcommand{\OtherTok}[1]{\textcolor[rgb]{0.56,0.35,0.01}{#1}}
\newcommand{\PreprocessorTok}[1]{\textcolor[rgb]{0.56,0.35,0.01}{\textit{#1}}}
\newcommand{\RegionMarkerTok}[1]{#1}
\newcommand{\SpecialCharTok}[1]{\textcolor[rgb]{0.00,0.00,0.00}{#1}}
\newcommand{\SpecialStringTok}[1]{\textcolor[rgb]{0.31,0.60,0.02}{#1}}
\newcommand{\StringTok}[1]{\textcolor[rgb]{0.31,0.60,0.02}{#1}}
\newcommand{\VariableTok}[1]{\textcolor[rgb]{0.00,0.00,0.00}{#1}}
\newcommand{\VerbatimStringTok}[1]{\textcolor[rgb]{0.31,0.60,0.02}{#1}}
\newcommand{\WarningTok}[1]{\textcolor[rgb]{0.56,0.35,0.01}{\textbf{\textit{#1}}}}
\usepackage{graphicx}
\makeatletter
\def\maxwidth{\ifdim\Gin@nat@width>\linewidth\linewidth\else\Gin@nat@width\fi}
\def\maxheight{\ifdim\Gin@nat@height>\textheight\textheight\else\Gin@nat@height\fi}
\makeatother
% Scale images if necessary, so that they will not overflow the page
% margins by default, and it is still possible to overwrite the defaults
% using explicit options in \includegraphics[width, height, ...]{}
\setkeys{Gin}{width=\maxwidth,height=\maxheight,keepaspectratio}
% Set default figure placement to htbp
\makeatletter
\def\fps@figure{htbp}
\makeatother
\setlength{\emergencystretch}{3em} % prevent overfull lines
\providecommand{\tightlist}{%
  \setlength{\itemsep}{0pt}\setlength{\parskip}{0pt}}
\setcounter{secnumdepth}{-\maxdimen} % remove section numbering
\ifLuaTeX
  \usepackage{selnolig}  % disable illegal ligatures
\fi

\title{Internal\_hypervolume\_structure}
\author{Patrick Alexander Walkden}
\date{2022-07-12}

\begin{document}
\maketitle

\hypertarget{analysis-of-the-impacts-of-land-use-on-the-internal-structure-of-avian-functional-trait-space}{%
\subsection{Analysis of the impacts of land use on the internal
structure of avian functional trait
space}\label{analysis-of-the-impacts-of-land-use-on-the-internal-structure-of-avian-functional-trait-space}}

This markdown re-visits the hole analysis that was somewhat abandoned in
the first pass. The goal was to capture changes in the internal
structure of trait spaces, which traditional methods and approaches
aren't typically able to do so. To reveal such impacts I aim to detect
``hole'' within the bounds of a convex hull surrounding the observed
hypervolumes. I defined holes in an analougous way to Blonder by
encasing the observed hypervolume in an convex hull and then comparing
the trait space to a ``baseline'' condition. Where we differ is that my
baseline is defined as a null expectation contingent on the observed
hypervolumes regional species pool, as opposed to all unoccupied trait
space being considered a ``hole''. Another difference can be seen in the
clustering approach to group empty space into distinct ``holes''.
Blonder uses distance based clustering approaches, which is poor for
detecting non-circular holes. I using density based clustering which
defines holes based on the number of points within a minimum distance to
each point. To accont for the sensitivity of hole detection to this
minimum distance I create hole profiles calculating the number of holes
and their size as this radius increases.

Intuitively, one would expect that bird communities in land uses of
increasing human influence would display a greater erosion of their
internal trait space, but we will see what we capture.

\hypertarget{load-in-data-packages-and-functions}{%
\subsection{Load in data, packages and
functions}\label{load-in-data-packages-and-functions}}

\begin{Shaded}
\begin{Highlighting}[]
\FunctionTok{rm}\NormalTok{(}\AttributeTok{list =} \FunctionTok{ls}\NormalTok{())}

\FunctionTok{require}\NormalTok{(mgcv) }\DocumentationTok{\#\# to perform GAMs}
\end{Highlighting}
\end{Shaded}

\begin{verbatim}
## Loading required package: mgcv
\end{verbatim}

\begin{verbatim}
## Warning: package 'mgcv' was built under R version 4.1.3
\end{verbatim}

\begin{verbatim}
## Loading required package: nlme
\end{verbatim}

\begin{verbatim}
## This is mgcv 1.8-40. For overview type 'help("mgcv-package")'.
\end{verbatim}

\begin{Shaded}
\begin{Highlighting}[]
\FunctionTok{require}\NormalTok{(tidyverse) }\DocumentationTok{\#\# for data  wrangling and piping}
\end{Highlighting}
\end{Shaded}

\begin{verbatim}
## Loading required package: tidyverse
\end{verbatim}

\begin{verbatim}
## -- Attaching packages --------------------------------------- tidyverse 1.3.1 --
\end{verbatim}

\begin{verbatim}
## v ggplot2 3.3.6     v purrr   0.3.4
## v tibble  3.1.7     v dplyr   1.0.9
## v tidyr   1.2.0     v stringr 1.4.0
## v readr   2.1.2     v forcats 0.5.1
\end{verbatim}

\begin{verbatim}
## Warning: package 'ggplot2' was built under R version 4.1.3
\end{verbatim}

\begin{verbatim}
## Warning: package 'tibble' was built under R version 4.1.3
\end{verbatim}

\begin{verbatim}
## Warning: package 'tidyr' was built under R version 4.1.3
\end{verbatim}

\begin{verbatim}
## Warning: package 'readr' was built under R version 4.1.3
\end{verbatim}

\begin{verbatim}
## Warning: package 'dplyr' was built under R version 4.1.3
\end{verbatim}

\begin{verbatim}
## -- Conflicts ------------------------------------------ tidyverse_conflicts() --
## x dplyr::collapse() masks nlme::collapse()
## x dplyr::filter()   masks stats::filter()
## x dplyr::lag()      masks stats::lag()
\end{verbatim}

\begin{Shaded}
\begin{Highlighting}[]
\FunctionTok{source}\NormalTok{(}\StringTok{"../Functions/TPD\_3D\_plots.R"}\NormalTok{) }\DocumentationTok{\#\# my very own TPD functions}
\end{Highlighting}
\end{Shaded}

\begin{verbatim}
## Loading required package: gstat
\end{verbatim}

\begin{verbatim}
## Warning: package 'gstat' was built under R version 4.1.3
\end{verbatim}

\begin{verbatim}
## Loading required package: sf
\end{verbatim}

\begin{verbatim}
## Warning: package 'sf' was built under R version 4.1.3
\end{verbatim}

\begin{verbatim}
## Linking to GEOS 3.9.1, GDAL 3.2.1, PROJ 7.2.1; sf_use_s2() is TRUE
\end{verbatim}

\begin{verbatim}
## Loading required package: ggpubr
\end{verbatim}

\begin{verbatim}
## Loading required package: magrittr
\end{verbatim}

\begin{verbatim}
## Warning: package 'magrittr' was built under R version 4.1.3
\end{verbatim}

\begin{verbatim}
## 
## Attaching package: 'magrittr'
\end{verbatim}

\begin{verbatim}
## The following object is masked from 'package:purrr':
## 
##     set_names
\end{verbatim}

\begin{verbatim}
## The following object is masked from 'package:tidyr':
## 
##     extract
\end{verbatim}

\begin{verbatim}
## Loading required package: rgl
\end{verbatim}

\begin{verbatim}
## Warning: package 'rgl' was built under R version 4.1.3
\end{verbatim}

\begin{verbatim}
## Loading required package: geometry
\end{verbatim}

\begin{verbatim}
## Warning: package 'geometry' was built under R version 4.1.3
\end{verbatim}

\begin{verbatim}
## Loading required package: fastcluster
\end{verbatim}

\begin{verbatim}
## 
## Attaching package: 'fastcluster'
\end{verbatim}

\begin{verbatim}
## The following object is masked from 'package:stats':
## 
##     hclust
\end{verbatim}

\begin{verbatim}
## Loading required package: Hmisc
\end{verbatim}

\begin{verbatim}
## Warning: package 'Hmisc' was built under R version 4.1.3
\end{verbatim}

\begin{verbatim}
## Loading required package: lattice
\end{verbatim}

\begin{verbatim}
## Loading required package: survival
\end{verbatim}

\begin{verbatim}
## Loading required package: Formula
\end{verbatim}

\begin{verbatim}
## 
## Attaching package: 'Hmisc'
\end{verbatim}

\begin{verbatim}
## The following objects are masked from 'package:dplyr':
## 
##     src, summarize
\end{verbatim}

\begin{verbatim}
## The following objects are masked from 'package:base':
## 
##     format.pval, units
\end{verbatim}

\begin{verbatim}
## Loading required package: magick
\end{verbatim}

\begin{verbatim}
## Linking to ImageMagick 6.9.12.3
## Enabled features: cairo, freetype, fftw, ghostscript, heic, lcms, pango, raw, rsvg, webp
## Disabled features: fontconfig, x11
\end{verbatim}

\begin{verbatim}
## Warning: package 'webshot2' was built under R version 4.1.3
\end{verbatim}

\begin{verbatim}
## Warning in gzfile(file, "rb"): cannot open compressed file 'Functions/
## TPD_colours.rds', probable reason 'No such file or directory'
\end{verbatim}

\begin{verbatim}
## Error in gzfile(file, "rb") : cannot open the connection
\end{verbatim}

\begin{verbatim}
## Loading required package: dbscan
\end{verbatim}

\begin{verbatim}
## Warning: package 'dbscan' was built under R version 4.1.3
\end{verbatim}

\begin{verbatim}
## Loading required package: fpc
\end{verbatim}

\begin{verbatim}
## Warning: package 'fpc' was built under R version 4.1.3
\end{verbatim}

\begin{verbatim}
## 
## Attaching package: 'fpc'
\end{verbatim}

\begin{verbatim}
## The following object is masked from 'package:dbscan':
## 
##     dbscan
\end{verbatim}

\begin{verbatim}
## Loading required package: factoextra
\end{verbatim}

\begin{verbatim}
## Warning: package 'factoextra' was built under R version 4.1.3
\end{verbatim}

\begin{verbatim}
## Welcome! Want to learn more? See two factoextra-related books at https://goo.gl/ve3WBa
\end{verbatim}

\begin{Shaded}
\begin{Highlighting}[]
\NormalTok{PREDICTS\_tpds }\OtherTok{\textless{}{-}}
  \FunctionTok{readRDS}\NormalTok{(}\StringTok{"../Outputs/PREDICTS\_sites\_tpds.rds"}\NormalTok{) }\CommentTok{\# morphometric TPDs of observed sites}

\NormalTok{PREDICTS\_randomisations }\OtherTok{\textless{}{-}}
  \FunctionTok{readRDS}\NormalTok{(}\StringTok{"../Outputs/randomisations\_TPD\_morpho.rds"}\NormalTok{) }\DocumentationTok{\#\# morphometric TPD of randomisised sites}




\NormalTok{PREDICTS\_full }\OtherTok{\textless{}{-}} \FunctionTok{readRDS}\NormalTok{(}\StringTok{"../Outputs/refined\_predicts.rds"}\NormalTok{)}

\NormalTok{PREDICTS }\OtherTok{\textless{}{-}}\NormalTok{ PREDICTS\_full }\SpecialCharTok{\%\textgreater{}\%}  \DocumentationTok{\#\# PREDICTS data}
\NormalTok{  dplyr}\SpecialCharTok{::}\FunctionTok{distinct}\NormalTok{(SSBS, Predominant\_habitat, Realm, SS) }\SpecialCharTok{\%\textgreater{}\%} \DocumentationTok{\#\# pull out land\_use type, Subregion, realm etc}
\NormalTok{  dplyr}\SpecialCharTok{::}\FunctionTok{mutate}\NormalTok{(}
    \AttributeTok{Predominant\_habitat =} \FunctionTok{ifelse}\NormalTok{(}
      \FunctionTok{grepl}\NormalTok{(}
\NormalTok{        Predominant\_habitat,}
        \AttributeTok{pattern =} \StringTok{"secondary"}\NormalTok{,}
        \AttributeTok{ignore.case =} \ConstantTok{TRUE}
\NormalTok{      ),}
      \StringTok{"Secondary vegetation"}\NormalTok{,}
      \FunctionTok{paste}\NormalTok{(Predominant\_habitat)}
\NormalTok{    ),}
    \AttributeTok{Predominant\_habitat =} \FunctionTok{ifelse}\NormalTok{(}
      \FunctionTok{grepl}\NormalTok{(}
\NormalTok{        Predominant\_habitat,}
        \AttributeTok{pattern =} \StringTok{"primary"}\NormalTok{,}
        \AttributeTok{ignore.case =} \ConstantTok{TRUE}
\NormalTok{      ),}
      \StringTok{"Primary vegetation"}\NormalTok{,}
      \FunctionTok{paste}\NormalTok{(Predominant\_habitat)}
\NormalTok{    )}
\NormalTok{  ) }\SpecialCharTok{\%\textgreater{}\%} \FunctionTok{data.frame}\NormalTok{() }\DocumentationTok{\#\# merge all secondary sites together}



\NormalTok{TPD\_land\_uses }\OtherTok{\textless{}{-}}
  \FunctionTok{data.frame}\NormalTok{(}\AttributeTok{SSBS =} \FunctionTok{names}\NormalTok{(PREDICTS\_tpds)) }\SpecialCharTok{\%\textgreater{}\%}\NormalTok{ dplyr}\SpecialCharTok{::}\FunctionTok{left\_join}\NormalTok{(PREDICTS, }\AttributeTok{by =} \StringTok{"SSBS"}\NormalTok{) }\SpecialCharTok{\%\textgreater{}\%}\NormalTok{ dplyr}\SpecialCharTok{::}\FunctionTok{filter}\NormalTok{(Predominant\_habitat }\SpecialCharTok{!=} \StringTok{"Cannot decide"}\NormalTok{)}
\DocumentationTok{\#\#\# check a table}
\FunctionTok{table}\NormalTok{(TPD\_land\_uses}\SpecialCharTok{$}\NormalTok{Predominant\_habitat, TPD\_land\_uses}\SpecialCharTok{$}\NormalTok{Realm)}
\end{Highlighting}
\end{Shaded}

\begin{verbatim}
##                       
##                        Afrotropic Australasia Indo-Malay Nearctic Neotropic
##   Cropland                     75           1         20       16        18
##   Pasture                      20          22          0       28        68
##   Plantation forest            89           3        119        3        32
##   Primary vegetation          163          57         78       85        86
##   Secondary vegetation         99          25         21        0        55
##   Urban                         9          11          4       29        40
##                       
##                        Palearctic
##   Cropland                     23
##   Pasture                      26
##   Plantation forest           161
##   Primary vegetation          106
##   Secondary vegetation        110
##   Urban                        30
\end{verbatim}

\begin{Shaded}
\begin{Highlighting}[]
\NormalTok{land\_uses }\OtherTok{\textless{}{-}}
  \FunctionTok{c}\NormalTok{(}
    \StringTok{"Primary vegetation"}\NormalTok{,}
    \StringTok{"Secondary vegetation"}\NormalTok{,}
    \StringTok{"Pasture"}\NormalTok{,}
    \StringTok{"Cropland"}\NormalTok{,}
    \StringTok{"Plantation forest"}\NormalTok{,}
    \StringTok{"Urban"}
\NormalTok{  )}
\NormalTok{realms }\OtherTok{\textless{}{-}}
  \FunctionTok{c}\NormalTok{(}\StringTok{"Neotropic"}\NormalTok{,}
    \StringTok{"Afrotropic"}\NormalTok{,}
    \StringTok{"Palearctic"}\NormalTok{,}
    \StringTok{"Nearctic"}\NormalTok{,}
    \StringTok{"Indo{-}Malay"}\NormalTok{,}
    \StringTok{"Australasia"}\NormalTok{)}


\NormalTok{land\_use\_colours }\OtherTok{\textless{}{-}}
  \FunctionTok{data.frame}\NormalTok{(}
    \AttributeTok{land\_use =}\NormalTok{ land\_uses,}
    \AttributeTok{colours =} \FunctionTok{c}\NormalTok{(}
      \StringTok{"chartreuse4"}\NormalTok{,}
      \StringTok{"olivedrab2"}\NormalTok{,}
      \StringTok{"\#EBF787"}\NormalTok{,}
      \StringTok{"\#E3D438"}\NormalTok{,}
      \StringTok{"springgreen2"}\NormalTok{,}
      \StringTok{"\#718879"}
\NormalTok{    )}
\NormalTok{  )}
\FunctionTok{rownames}\NormalTok{(land\_use\_colours) }\OtherTok{\textless{}{-}}\NormalTok{ land\_uses}
\end{Highlighting}
\end{Shaded}

\hypertarget{first-things-first-calculate-the-hole-statistics}{%
\subsection{First things first: calculate the hole
statistics}\label{first-things-first-calculate-the-hole-statistics}}

This analysis will be based on the Land-Use Realm system hypervolumes to
be able to more clearly reveal geomertic patterns, which at the site
level is just too noisy and suffers from data limitations.

\hypertarget{trimming-things-down-a-little}{%
\subsection{Trimming things down a
little}\label{trimming-things-down-a-little}}

Now need to exclude some Land-Use Realm hypervolumes that were too data
poor to get reliable measures. Also to facilitate modelling I scale some
variables.

\begin{Shaded}
\begin{Highlighting}[]
\NormalTok{realm\_level\_hole\_frame}\SpecialCharTok{$}\NormalTok{land\_use }\OtherTok{\textless{}{-}}
  \FunctionTok{factor}\NormalTok{(}
\NormalTok{    realm\_level\_hole\_frame}\SpecialCharTok{$}\NormalTok{land\_use,}
    \AttributeTok{levels =} \FunctionTok{c}\NormalTok{(}
      \StringTok{"Primary vegetation"}\NormalTok{,}
      \StringTok{"Secondary vegetation"}\NormalTok{,}
      \StringTok{"Plantation forest"}\NormalTok{,}
      \StringTok{"Pasture"}\NormalTok{,}
      \StringTok{"Cropland"}\NormalTok{,}
      \StringTok{"Urban"}
\NormalTok{    )}
\NormalTok{  )}


\NormalTok{modelling\_hole\_dataframe }\OtherTok{\textless{}{-}}
\NormalTok{  realm\_level\_hole\_frame[}\SpecialCharTok{{-}}\FunctionTok{which}\NormalTok{(}
\NormalTok{    realm\_level\_hole\_frame}\SpecialCharTok{$}\NormalTok{land\_use }\SpecialCharTok{==} \StringTok{"Urban"} \SpecialCharTok{\&}
\NormalTok{      realm\_level\_hole\_frame}\SpecialCharTok{$}\NormalTok{realm }\SpecialCharTok{==} \StringTok{"Neotropic"} \SpecialCharTok{|}
\NormalTok{      realm\_level\_hole\_frame}\SpecialCharTok{$}\NormalTok{land\_use }\SpecialCharTok{==} \StringTok{"Secondary vegetation"} \SpecialCharTok{\&}
\NormalTok{      realm\_level\_hole\_frame}\SpecialCharTok{$}\NormalTok{realm }\SpecialCharTok{==} \StringTok{"Nearctic"} \SpecialCharTok{|}
\NormalTok{      realm\_level\_hole\_frame}\SpecialCharTok{$}\NormalTok{land\_use }\SpecialCharTok{==} \StringTok{"Plantation forest"} \SpecialCharTok{\&}
\NormalTok{      realm\_level\_hole\_frame}\SpecialCharTok{$}\NormalTok{realm }\SpecialCharTok{==} \StringTok{"Nearctic"} \SpecialCharTok{|}
\NormalTok{      realm\_level\_hole\_frame}\SpecialCharTok{$}\NormalTok{land\_use }\SpecialCharTok{==} \StringTok{"Urban"} \SpecialCharTok{\&}
\NormalTok{      realm\_level\_hole\_frame}\SpecialCharTok{$}\NormalTok{realm }\SpecialCharTok{==} \StringTok{"Australasia"} \SpecialCharTok{|}
\NormalTok{      realm\_level\_hole\_frame}\SpecialCharTok{$}\NormalTok{land\_use }\SpecialCharTok{==} \StringTok{"Plantation forest"} \SpecialCharTok{\&}
\NormalTok{      realm\_level\_hole\_frame}\SpecialCharTok{$}\NormalTok{realm }\SpecialCharTok{==} \StringTok{"Australasia"} \SpecialCharTok{|}
\NormalTok{      realm\_level\_hole\_frame}\SpecialCharTok{$}\NormalTok{land\_use }\SpecialCharTok{==} \StringTok{"Cropland"} \SpecialCharTok{\&}
\NormalTok{      realm\_level\_hole\_frame}\SpecialCharTok{$}\NormalTok{realm }\SpecialCharTok{==} \StringTok{"Australasia"} \SpecialCharTok{|}
\NormalTok{      realm\_level\_hole\_frame}\SpecialCharTok{$}\NormalTok{land\_use }\SpecialCharTok{==} \StringTok{"Pasture"} \SpecialCharTok{\&}
\NormalTok{      realm\_level\_hole\_frame}\SpecialCharTok{$}\NormalTok{realm }\SpecialCharTok{==} \StringTok{"Indo{-}Malay"} \SpecialCharTok{|}
\NormalTok{      realm\_level\_hole\_frame}\SpecialCharTok{$}\NormalTok{land\_use }\SpecialCharTok{==} \StringTok{"Urban"} \SpecialCharTok{\&}
\NormalTok{      realm\_level\_hole\_frame}\SpecialCharTok{$}\NormalTok{realm }\SpecialCharTok{==} \StringTok{"Indo{-}Malay"}
\NormalTok{  ), ]}


\DocumentationTok{\#\# Let\textquotesingle{}s have a look at the total hole volume}


\FunctionTok{hist}\NormalTok{(}\FunctionTok{scale}\NormalTok{(modelling\_hole\_dataframe}\SpecialCharTok{$}\NormalTok{total\_hole\_volume), }\AttributeTok{breaks =} \DecValTok{20}\NormalTok{)}
\end{Highlighting}
\end{Shaded}

\includegraphics{Internal_structure_of_trait_space_files/figure-latex/trimming things down-1.pdf}

\begin{Shaded}
\begin{Highlighting}[]
\FunctionTok{hist}\NormalTok{(}\FunctionTok{scale}\NormalTok{(}\FunctionTok{sqrt}\NormalTok{(modelling\_hole\_dataframe}\SpecialCharTok{$}\NormalTok{total\_hole\_volume)), }\AttributeTok{breaks =} \DecValTok{20}\NormalTok{)}
\end{Highlighting}
\end{Shaded}

\includegraphics{Internal_structure_of_trait_space_files/figure-latex/trimming things down-2.pdf}

\begin{Shaded}
\begin{Highlighting}[]
\FunctionTok{hist}\NormalTok{(}\FunctionTok{scale}\NormalTok{(}\FunctionTok{log}\NormalTok{(modelling\_hole\_dataframe}\SpecialCharTok{$}\NormalTok{total\_hole\_volume)), }\AttributeTok{breaks =} \DecValTok{20}\NormalTok{)}
\end{Highlighting}
\end{Shaded}

\includegraphics{Internal_structure_of_trait_space_files/figure-latex/trimming things down-3.pdf}

\begin{Shaded}
\begin{Highlighting}[]
\DocumentationTok{\#\#\# out of these three I think the sqrt transformation is the best }


\DocumentationTok{\#\# looks okay let\textquotesingle{}s just scale it to have a standard deviation of 1 and a mean of 0}

\NormalTok{modelling\_hole\_dataframe}\SpecialCharTok{$}\NormalTok{total\_hole\_volume }\OtherTok{\textless{}{-}} \FunctionTok{scale}\NormalTok{(}\FunctionTok{sqrt}\NormalTok{(modelling\_hole\_dataframe}\SpecialCharTok{$}\NormalTok{total\_hole\_volume))[,}\DecValTok{1}\NormalTok{]}


\DocumentationTok{\#\# the size of the observed hypervolume }

\FunctionTok{hist}\NormalTok{(}\FunctionTok{scale}\NormalTok{(modelling\_hole\_dataframe}\SpecialCharTok{$}\NormalTok{hypervolume\_occupancy))}
\end{Highlighting}
\end{Shaded}

\includegraphics{Internal_structure_of_trait_space_files/figure-latex/trimming things down-4.pdf}

\begin{Shaded}
\begin{Highlighting}[]
\DocumentationTok{\#\# okay maybe could use a transformation}


\FunctionTok{hist}\NormalTok{(}\FunctionTok{scale}\NormalTok{(}\FunctionTok{log}\NormalTok{(modelling\_hole\_dataframe}\SpecialCharTok{$}\NormalTok{hypervolume\_occupancy)))}
\end{Highlighting}
\end{Shaded}

\includegraphics{Internal_structure_of_trait_space_files/figure-latex/trimming things down-5.pdf}

\begin{Shaded}
\begin{Highlighting}[]
\DocumentationTok{\#\# log probably looks best }


\NormalTok{modelling\_hole\_dataframe}\SpecialCharTok{$}\NormalTok{hypervolume\_occupancy }\OtherTok{\textless{}{-}} \FunctionTok{scale}\NormalTok{(}\FunctionTok{log}\NormalTok{(modelling\_hole\_dataframe}\SpecialCharTok{$}\NormalTok{hypervolume\_occupancy))[,}\DecValTok{1}\NormalTok{]}


\FunctionTok{source}\NormalTok{(}\StringTok{"https://highstat.com/Books/Book2/HighstatLibV10.R"}\NormalTok{)}

\FunctionTok{corvif}\NormalTok{(modelling\_hole\_dataframe[,}\FunctionTok{c}\NormalTok{(}\StringTok{"total\_hole\_volume"}\NormalTok{,}\StringTok{"hypervolume\_occupancy"}\NormalTok{,}\StringTok{"land\_use"}\NormalTok{)])}
\end{Highlighting}
\end{Shaded}

\begin{verbatim}
## 
## 
## Variance inflation factors
## 
##                           GVIF Df GVIF^(1/2Df)
## total_hole_volume     2.287586  1     1.512477
## hypervolume_occupancy 2.826103  1     1.681102
## land_use              1.732807  5     1.056513
\end{verbatim}

\begin{Shaded}
\begin{Highlighting}[]
\DocumentationTok{\#\# brilliant this shows that there is no multi{-}colinearity so good to go ahead with the modelling}
\end{Highlighting}
\end{Shaded}

\hypertarget{pre-modelling}{%
\subsection{Pre-modelling}\label{pre-modelling}}

Before I start modelling I would like to look at what sort of
relationship the variables have with each other, for example the radius
of hole detection is not likely to have a linear relationship with total
hole volume as it will initially increase sharply with radius but then
plateau when all unoccupied space is considered to be part of a hole.

So to have a look at this we

\begin{Shaded}
\begin{Highlighting}[]
\DocumentationTok{\#\#\# relationship between radius and hole volume }


\NormalTok{radius\_hole\_volume }\OtherTok{\textless{}{-}} \FunctionTok{ggplot}\NormalTok{(}\AttributeTok{data =}\NormalTok{ modelling\_hole\_dataframe, }\FunctionTok{aes}\NormalTok{(}\AttributeTok{x =}\NormalTok{ radius, }\AttributeTok{y =}\NormalTok{ total\_hole\_volume, }\AttributeTok{colour =}\NormalTok{ land\_use))}\SpecialCharTok{+}
  \FunctionTok{geom\_point}\NormalTok{() }\SpecialCharTok{+}
  \FunctionTok{facet\_grid}\NormalTok{(}\SpecialCharTok{\textasciitilde{}}\NormalTok{realm)}


\FunctionTok{plot}\NormalTok{(radius\_hole\_volume)}
\end{Highlighting}
\end{Shaded}

\includegraphics{Internal_structure_of_trait_space_files/figure-latex/variable relationships-1.pdf}

\begin{Shaded}
\begin{Highlighting}[]
\DocumentationTok{\#\# right this looks non{-}linear so looking like we are going to need a non{-}linear modelling approach I\textquotesingle{}m thinkin generalised linear models (GAMs)}


\NormalTok{hole\_volume\_occupancy }\OtherTok{\textless{}{-}} \FunctionTok{ggplot}\NormalTok{(}\AttributeTok{data =}\NormalTok{ modelling\_hole\_dataframe, }\FunctionTok{aes}\NormalTok{(}\AttributeTok{x =}\NormalTok{ hypervolume\_occupancy, }\AttributeTok{y =}\NormalTok{ total\_hole\_volume, }\AttributeTok{colour =}\NormalTok{ land\_use)) }\SpecialCharTok{+}
  \FunctionTok{geom\_point}\NormalTok{()}


\FunctionTok{plot}\NormalTok{(hole\_volume\_occupancy)}
\end{Highlighting}
\end{Shaded}

\includegraphics{Internal_structure_of_trait_space_files/figure-latex/variable relationships-2.pdf}

\begin{Shaded}
\begin{Highlighting}[]
\DocumentationTok{\#\# Just looking at the top values here the realtionship is looking a lot more linear with increase observed hypervolume size the larger the observed hole may be.}
\end{Highlighting}
\end{Shaded}

\hypertarget{modelling}{%
\subsection{Modelling}\label{modelling}}

Considering the relationships between the variables I think the model to
explore the influence of land use on the internal structure of avian
trait space will take the form of\ldots{}

A generalised additive model (GAM) taking total hole volume as a
function of land use, the detection radius of hole as a smoothed
variable, observed hypervolume size as a linear variable and the
interaction of both with land use.

\begin{Shaded}
\begin{Highlighting}[]
\NormalTok{hole\_volume\_gam }\OtherTok{\textless{}{-}}\NormalTok{ mgcv}\SpecialCharTok{::}\FunctionTok{gam}\NormalTok{(}
\NormalTok{  total\_hole\_volume }\SpecialCharTok{\textasciitilde{}} \FunctionTok{s}\NormalTok{(radius) }\SpecialCharTok{+}
    \FunctionTok{s}\NormalTok{(radius, }\AttributeTok{by =}\NormalTok{ land\_use) }\SpecialCharTok{+}
\NormalTok{    hypervolume\_occupancy }\SpecialCharTok{+}
    \CommentTok{\#te(hypervolume\_occupancy,radius)+}
\NormalTok{    hypervolume\_occupancy}\SpecialCharTok{:}\NormalTok{land\_use  }\SpecialCharTok{+}
\NormalTok{    land\_use,}
  \AttributeTok{data =}\NormalTok{ modelling\_hole\_dataframe,}
  \AttributeTok{method =} \StringTok{"REML"}
\NormalTok{)}


\FunctionTok{summary}\NormalTok{(hole\_volume\_gam)}
\end{Highlighting}
\end{Shaded}

\begin{verbatim}
## 
## Family: gaussian 
## Link function: identity 
## 
## Formula:
## total_hole_volume ~ s(radius) + s(radius, by = land_use) + hypervolume_occupancy + 
##     hypervolume_occupancy:land_use + land_use
## 
## Parametric coefficients:
##                                                    Estimate Std. Error t value
## (Intercept)                                        -0.49343    0.02645 -18.652
## hypervolume_occupancy                               0.75796    0.02658  28.519
## land_useSecondary vegetation                        0.57556    0.03563  16.152
## land_usePlantation forest                           0.69090    0.05214  13.250
## land_usePasture                                     0.69337    0.03478  19.935
## land_useCropland                                    0.67763    0.03425  19.786
## land_useUrban                                       0.38402    0.15487   2.480
## hypervolume_occupancy:land_useSecondary vegetation -0.20490    0.03922  -5.224
## hypervolume_occupancy:land_usePlantation forest    -0.05633    0.04762  -1.183
## hypervolume_occupancy:land_usePasture               0.01062    0.03504   0.303
## hypervolume_occupancy:land_useCropland              0.10885    0.03948   2.757
## hypervolume_occupancy:land_useUrban                -0.00640    0.05427  -0.118
##                                                    Pr(>|t|)    
## (Intercept)                                         < 2e-16 ***
## hypervolume_occupancy                               < 2e-16 ***
## land_useSecondary vegetation                        < 2e-16 ***
## land_usePlantation forest                           < 2e-16 ***
## land_usePasture                                     < 2e-16 ***
## land_useCropland                                    < 2e-16 ***
## land_useUrban                                       0.01325 *  
## hypervolume_occupancy:land_useSecondary vegetation 1.96e-07 ***
## hypervolume_occupancy:land_usePlantation forest     0.23694    
## hypervolume_occupancy:land_usePasture               0.76182    
## hypervolume_occupancy:land_useCropland              0.00589 ** 
## hypervolume_occupancy:land_useUrban                 0.90615    
## ---
## Signif. codes:  0 '***' 0.001 '**' 0.01 '*' 0.05 '.' 0.1 ' ' 1
## 
## Approximate significance of smooth terms:
##                                              edf    Ref.df       F  p-value    
## s(radius)                              8.7588867 8.9659331 157.147  < 2e-16 ***
## s(radius):land_usePrimary vegetation   1.0006844 1.0011694   1.190  0.27559    
## s(radius):land_useSecondary vegetation 2.9146880 3.6236653   2.815  0.02999 *  
## s(radius):land_usePlantation forest    0.0004687 0.0009231   0.044  0.99490    
## s(radius):land_usePasture              1.5149511 1.8577096   0.414  0.56726    
## s(radius):land_useCropland             3.2442492 4.0353832   3.970  0.00285 ** 
## s(radius):land_useUrban                3.8614894 4.5286250   7.811 7.38e-06 ***
## ---
## Signif. codes:  0 '***' 0.001 '**' 0.01 '*' 0.05 '.' 0.1 ' ' 1
## 
## Rank: 74/75
## R-sq.(adj) =  0.832   Deviance explained = 83.5%
## -REML = 1018.4  Scale est. = 0.1682    n = 1802
\end{verbatim}

\hypertarget{hole-volume-summary}{%
\subsection{Hole volume summary}\label{hole-volume-summary}}

This model shows that holes are likely to larger is avian trait spaces
of bird communities in all other land uses compared to primary
vegetation, significantly so in all land uses other than urban. The size
of the observed hypervolume and hole detection radius also have
significant effects on the total hole volume, unsurprisingly the larger
the observed hypervolume and detection radius the larger the holes.
Their interactions with land use are also significant.

Some interesting interactions going on here with urban holes becoming
smaller as observed hypervolume size increases, however observed
hypervolume size for urban LURs are all typically very small.

But, lets get some visualisations to aid us in the interpretations of
the model. first let's have a look at the effect of the hole detection
radius on the total hole volume for each land use. To predict the model
I have kept the hypervolume occupancy as the median hypervolume
occupancy observed across all viable LUR combinations.

\begin{Shaded}
\begin{Highlighting}[]
\NormalTok{gamm\_plot\_radius }\OtherTok{\textless{}{-}} \FunctionTok{c}\NormalTok{()}



\ControlFlowTok{for}\NormalTok{ (land\_use }\ControlFlowTok{in}\NormalTok{ land\_uses) \{}
  
  
\NormalTok{  land\_use\_data }\OtherTok{\textless{}{-}}
\NormalTok{    modelling\_hole\_dataframe[modelling\_hole\_dataframe}\SpecialCharTok{$}\NormalTok{land\_use }\SpecialCharTok{==}\NormalTok{ land\_use, }\StringTok{"radius"}\NormalTok{]}
  
  \CommentTok{\#hyper\_size \textless{}{-} range(unique(realm\_level\_hole\_frame\_2[realm\_level\_hole\_frame\_2$land\_use == land\_use, "hypervolume\_occupancy"]))}
  \CommentTok{\#hyper\_size \textless{}{-} seq(hyper\_size[1],hyper\_size[2], length.out = 100)}
  
  
  \ControlFlowTok{for}\NormalTok{ (i }\ControlFlowTok{in}\NormalTok{ land\_use\_data) \{}
   \CommentTok{\#   median\_volume \textless{}{-} median(unique(modelling\_hole\_dataframe[modelling\_hole\_dataframe$land\_use == land\_use,"hypervolume\_occupancy"]))}
\NormalTok{  median\_volume }\OtherTok{\textless{}{-}}
      \FunctionTok{median}\NormalTok{(}\FunctionTok{unique}\NormalTok{(modelling\_hole\_dataframe}\SpecialCharTok{$}\NormalTok{hypervolume\_occupancy))}
    
    
\NormalTok{    data }\OtherTok{\textless{}{-}} \FunctionTok{data.frame}\NormalTok{(}
      \AttributeTok{radius =}\NormalTok{ i,}
      \AttributeTok{land\_use =}\NormalTok{ land\_use,}
      \FunctionTok{predict}\NormalTok{(}
\NormalTok{        hole\_volume\_gam,}
        \AttributeTok{newdata =} \FunctionTok{data.frame}\NormalTok{(}
          \AttributeTok{radius =}\NormalTok{ i,}
          \AttributeTok{land\_use =}\NormalTok{ land\_use,}
          \AttributeTok{hypervolume\_occupancy =}\NormalTok{ median\_volume}
\NormalTok{        ),}
        \AttributeTok{se.fit =} \ConstantTok{TRUE}
\NormalTok{      )}
\NormalTok{    )}
    
\NormalTok{    gamm\_plot\_radius }\OtherTok{\textless{}{-}} \FunctionTok{rbind}\NormalTok{(gamm\_plot\_radius, data)}
\NormalTok{  \}}
\NormalTok{\}}


\NormalTok{gamm\_plot\_radius}\SpecialCharTok{$}\NormalTok{land\_use }\OtherTok{\textless{}{-}}
  \FunctionTok{factor}\NormalTok{(}
\NormalTok{    gamm\_plot\_radius}\SpecialCharTok{$}\NormalTok{land\_use,}
    \AttributeTok{levels =} \FunctionTok{c}\NormalTok{(}
      \StringTok{"Primary vegetation"}\NormalTok{,}
      \StringTok{"Secondary vegetation"}\NormalTok{,}
      \StringTok{"Pasture"}\NormalTok{,}
      \StringTok{"Cropland"}\NormalTok{,}
      \StringTok{"Plantation forest"}\NormalTok{,}
      \StringTok{"Urban"}
\NormalTok{    )}
\NormalTok{  )}


\NormalTok{gamm\_plot\_radius}\SpecialCharTok{$}\NormalTok{upper }\OtherTok{\textless{}{-}}
\NormalTok{  gamm\_plot\_radius}\SpecialCharTok{$}\NormalTok{fit }\SpecialCharTok{+}\NormalTok{ gamm\_plot\_radius}\SpecialCharTok{$}\NormalTok{se.fit}
\NormalTok{gamm\_plot\_radius}\SpecialCharTok{$}\NormalTok{lower }\OtherTok{\textless{}{-}}
\NormalTok{  gamm\_plot\_radius}\SpecialCharTok{$}\NormalTok{fit }\SpecialCharTok{{-}}\NormalTok{ gamm\_plot\_radius}\SpecialCharTok{$}\NormalTok{se.fit}



\NormalTok{test }\OtherTok{\textless{}{-}}
  \FunctionTok{ggplot}\NormalTok{(}\AttributeTok{data =}\NormalTok{ gamm\_plot\_radius, }\FunctionTok{aes}\NormalTok{(}\AttributeTok{x =}\NormalTok{ radius, }\AttributeTok{y =}\NormalTok{ fit, }\AttributeTok{group =}\NormalTok{ land\_use)) }\SpecialCharTok{+}
  \CommentTok{\#geom\_point(data = realm\_level\_radius\_data, aes( x = radius, y = mean\_hole\_volume, colour = land\_use), alpha = 0.7, size = 3) +}
  \FunctionTok{geom\_line}\NormalTok{(}\AttributeTok{size =} \FloatTok{1.5}\NormalTok{, }\AttributeTok{show.legend =} \ConstantTok{FALSE}\NormalTok{,}\FunctionTok{aes}\NormalTok{(}\AttributeTok{colour =}\NormalTok{ land\_use)) }\SpecialCharTok{+}
  \FunctionTok{geom\_ribbon}\NormalTok{(}\FunctionTok{aes}\NormalTok{(}
    \AttributeTok{ymin =}\NormalTok{ lower,}
    \AttributeTok{ymax =}\NormalTok{ upper,}
    \AttributeTok{fill =}\NormalTok{ land\_use}
\NormalTok{  ),}
  \AttributeTok{alpha =} \FloatTok{0.5}\NormalTok{, }\AttributeTok{show.legend =} \ConstantTok{FALSE}\NormalTok{) }\SpecialCharTok{+}
  \FunctionTok{scale\_colour\_manual}\NormalTok{(}\AttributeTok{values =}\NormalTok{ land\_use\_colours[}\FunctionTok{as.character}\NormalTok{(}\FunctionTok{unique}\NormalTok{(gamm\_plot\_radius}\SpecialCharTok{$}\NormalTok{land\_use)), }\StringTok{"colours"}\NormalTok{]) }\SpecialCharTok{+}
  \FunctionTok{scale\_fill\_manual}\NormalTok{(}\AttributeTok{values =}\NormalTok{ land\_use\_colours[}\FunctionTok{as.character}\NormalTok{(}\FunctionTok{unique}\NormalTok{(gamm\_plot\_radius}\SpecialCharTok{$}\NormalTok{land\_use)), }\StringTok{"colours"}\NormalTok{]) }\SpecialCharTok{+}
  \FunctionTok{theme}\NormalTok{(}\AttributeTok{panel.background =} \FunctionTok{element\_rect}\NormalTok{(}\AttributeTok{fill =} \StringTok{\textquotesingle{}grey\textquotesingle{}}\NormalTok{, }\AttributeTok{color =} \StringTok{\textquotesingle{}grey\textquotesingle{}}\NormalTok{),}
\AttributeTok{panel.grid.major =} \FunctionTok{element\_line}\NormalTok{(}\AttributeTok{color =} \StringTok{\textquotesingle{}grey\textquotesingle{}}\NormalTok{),}
\AttributeTok{panel.grid.minor =} \FunctionTok{element\_line}\NormalTok{(}\AttributeTok{color =} \StringTok{\textquotesingle{}grey\textquotesingle{}}\NormalTok{)) }\SpecialCharTok{+}
  \FunctionTok{xlab}\NormalTok{(}\StringTok{"Radius"}\NormalTok{) }\SpecialCharTok{+}
  \FunctionTok{ylab}\NormalTok{(}\StringTok{"Total Hole Volume"}\NormalTok{) }\SpecialCharTok{+}
  \FunctionTok{ylim}\NormalTok{(}\SpecialCharTok{{-}}\DecValTok{4}\NormalTok{,}\DecValTok{3}\NormalTok{) }


\FunctionTok{plot}\NormalTok{(test)}
\end{Highlighting}
\end{Shaded}

\includegraphics{Internal_structure_of_trait_space_files/figure-latex/observing GAM results-1.pdf}

This plot looks encouraging that at the median observed hypervolume size
that compared to primary vegetation all other land uses are predicted to
have larger holes at all hole detection radius values. Interestingly
again Urban bird communities are predicted to have the second smallest
holes at the median hypervolume occupancy, however, this may not be too
unexpected as urban Land-Use Realm systems typically have small observed
hypervolume sizes that may be susceptible to functional homogenisation
and these results may be indicative of that.

\includegraphics{"../Figures/Animations/radius_hole_volume/radius_hole_GIF.gif"}

\end{document}
